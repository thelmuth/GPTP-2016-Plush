%%%%%%%%%%%%%%%%%%%% author.tex %%%%%%%%%%%%%%%%%%%%%%%%%%%%%%%%%%%
%
%%%%%%%%%%%%%%%% Springer %%%%%%%%%%%%%%%%%%%%%%%%%%%%%%%%%%

\title*{Plush: Linear Genomes for PushGP}
% Use \titlerunning{Short Title} for an abbreviated version of
% your contribution title if the original one is too long
\author{Tom, Lee, Nic, Bill Tozier, Saul}
% Use \authorrunning{Short Title} for an abbreviated version of
% your contribution title if the original one is too long
\institute{Name of First Author \at Name, Address of Institute
\and Name of Second Author \at Name, Address of Institute}

\maketitle

\abstract{Each chapter should be preceded by an abstract (10--15 lines long) that summarizes the content.}

\begin{keywords}
keywords to your chapter, these words should also be indexed, do indexing later
\end{keywords}
\index{keywords to your chapter}
\index{these words should also be indexed}

\section{Introduction}
\label{Introduction}

Intro goes here. The overall story is that linear representations allows us to use clean uniform genetic operators and to get automatically hierarchical programs through automatic parentheses attached to specific instructions.



\section{Brief History of Push}


\section{Related Representations}



\section{Instruction-Based Hierarchy}

Here we talk about automatic parentheses.


\subsection{Experiment and Results}


\section{Uniform Genetic Operators}

Here we talk about uniform genetic operators, and how tree-based operators do bad things. Heavy citation of the ULTRA paper \citep{Spector:2013:GPTP}.

Is uniform mutation most important?


\subsection{Experiment and Results}








\begin{acknowledgement}
Go here.
\end{acknowledgement}
%


\bibliographystyle{spbasic}
\bibliography{gp-bibliography,helmuth}
