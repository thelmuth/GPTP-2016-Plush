%%%%%%%%%%%%%%%%%%%% author.tex %%%%%%%%%%%%%%%%%%%%%%%%%%%%%%%%%%%
%
%%%%%%%%%%%%%%%% Springer %%%%%%%%%%%%%%%%%%%%%%%%%%%%%%%%%%

\title*{Plush: Linear Genomes for PushGP}
% Use \titlerunning{Short Title} for an abbreviated version of
% your contribution title if the original one is too long
\author{Tom, Lee, Nic, Bill Tozier, Saul}
% Use \authorrunning{Short Title} for an abbreviated version of
% your contribution title if the original one is too long
\institute{Name of First Author \at Name, Address of Institute
\and Name of Second Author \at Name, Address of Institute}

\maketitle

\abstract{Each chapter should be preceded by an abstract (10--15 lines long) that summarizes the content.}

\begin{keywords}
keywords to your chapter, these words should also be indexed, do indexing later
\end{keywords}
\index{keywords to your chapter}
\index{these words should also be indexed}

\section{Introduction}
\label{Introduction}

Intro goes here. The overall story is that linear representations allows us to use clean uniform genetic operators and to get automatically hierarchical programs through automatic parentheses attached to specific instructions.



\section{Brief History of Push}



\section{Plush}



\section{Instruction-Based Hierarchy}

Here we talk about automatic parentheses.


\subsection{Experiment and Results}


\begin{table}
\centering
\caption{Number of successes, out of 100 runs, using either Plush (which automatically adds parentheses after specific instructions) and a version of Plush in which automatic parenthesizing was turned off. In the latter, instructions were inserted into the instruction set that simply opened parentheses. THIS CAPTION IS NOT VERY GOOD.}
\label{no-auto-parens-experiment}       % Give a unique label
%
% Follow this input for your own table layout
%
\begin{tabular}{l r r}
\hline\noalign{\smallskip}
Problem                    & Plush & ~No Auto-Parens \\
\noalign{\smallskip}\svhline\noalign{\smallskip}
Replace Space With Newline &  51 & 51 \\
Negative To Zero           &  45 & 34 \\
X-Word Lines               &   8 &  0 \\
Count Odds                 &   8 &  5 \\
\noalign{\smallskip}\hline\noalign{\smallskip}
\end{tabular}
\end{table}

\section{Uniform Genetic Operators}

Here we talk about uniform genetic operators, and how tree-based operators do bad things. Heavy citation of the ULTRA paper \citep{Spector:2013:GPTP}.

Is uniform mutation most important?


\subsection{Experiment and Results}



\begin{table}
\centering
\caption{Genetic operator combinations experimented with here. ``Alt.'' = alternation, ``Uni. Mut.'' = uniform mutation, ``Close Mut.'' = close mutation, and ``Alt. + Uni. Mut.'' = alternation followed by uniform mutation.}
\label{genetic-opeartor-combinations}
\begin{tabular}{ll llll}
\hline\noalign{\smallskip}
Designation & Description & Alt. & Uni. Mut. & Close Mut. & Alt. + Uni. Mut. \\
\noalign{\smallskip}\svhline\noalign{\smallskip}
REG & Regular Operators &  0.2 &  0.2 &  0.1 &  0.5  \\
NCM & No Close Mut.  &  0.22 &  0.22 &  0 &  0.56  \\
NUM & No Uni. Mut. &  0.9 &  0 &  0.1 &  0  \\
NA  & No Alt. &  0 &  0.9 &  0.1 &  0  \\
OUM & Onlt Uni. Mut. &  0 &  1.0 &  0 &  0  \\
\noalign{\smallskip}\hline\noalign{\smallskip}
\end{tabular}
\end{table}


\begin{table}
\centering
\caption{Genetic operators experiment.}
\label{genetic-opeartor-results}
\begin{tabular}{l r r r r r}
\hline\noalign{\smallskip}
Problem                    & REG & NA & NCM & NUM & OUM \\
\noalign{\smallskip}\svhline\noalign{\smallskip}
Replace Space With Newline &  51 & 55 &  50 &  24 &   x \\
Syllables                  &  18 &  9 &  20 &   7 &   x \\
Negative To Zero           &  45 & 46 &  41 &  11 &   x \\
X-Word Lines               &   8 &  1 &  12 &   0 &   x \\
Count Odds                 &   8 &  6 &   5 &   0 &   x \\
\noalign{\smallskip}\hline\noalign{\smallskip}
\end{tabular}
\end{table}



\section{Related Representations}







\begin{acknowledgement}
Go here.
\end{acknowledgement}
%


\bibliographystyle{spbasic}
\bibliography{gp-bibliography,helmuth}
